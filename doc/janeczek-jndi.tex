\documentclass[11pt,a4paper]{article}
\usepackage[utf8]{inputenc}
\usepackage{amsmath}
\usepackage{amsfonts}
\usepackage{amssymb}
\usepackage{graphicx}
\usepackage[left=2cm,right=2cm,top=2cm,bottom=2cm]{geometry}
\usepackage{authblk}
\title{Java Naming and Directory Interface(JNDI)}
\author{Janeczek Christian} 
\date{\today{}, Vienna}

\begin{document}

\maketitle
\newpage
\tableofcontents
\newpage

\section{Task Description}
Follow the introduction and instructions of the Java Naming and Directory Interface (JNDI) described in this tutorial \textbf{https://docs.oracle.com/javase/tutorial/jndi/TOC.html}. \newline

\noindent Setup your own or use an pre-existing naming service and implement following operations:
\begin{itemize}
	\item Lookup an Object (1)
	\item List the Context (1)
	\item Add, Replace or Remove a Binding (2)
	\item Rename (1)
	\item Create and Destroy Subcontexts (1)
	\item Attribute Names (1)
	\item Read Attributes (1)
	\item Modify Attributes (1)
	\item Add, Replace Bindings with Attributes (2)
	\item Basic Search (1)
	\item Filters (1)
	\item Scope (1)
	\item Result Count (1)
	\item Time Limit (1)
\end{itemize}
 
\noindent Create a protocol in which you describe 1) your \textbf{installation steps}, 2) \textbf{source code snippets} used to perform the naming operations and 3) \textbf{result} of each operation. Pack the protocol and sources into a JAR file and upload it here. \newline

\noindent Size of Group: 2 persons
(if you work without a team member then I will recommend to use the name service of another group)
\newpage

\section{Requirements}
\subsection{Installation Steps}
\subsection{Code Snippets for various Naming Operations}
\subsection{Result of each Operation}
\subsection{Effort Estimation + Actually Needed Time}
\newpage

\section{Functionality}
\subsection{Installation steps}
Thankfully enough the \textit{Java Naming and Directory Interface} is already included in the Java SE Platform. The applets, which will be written in Java are then being used by any Java-compatbile Web browser, such as Firefox, Google Chrome, Safari, etc.


\end{document}
